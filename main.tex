\documentclass[10pt]{article}
\usepackage[utf8]{inputenc}
\usepackage{graphicx}
\usepackage{hyperref}
\usepackage{fontenc}
\usepackage{mathptmx}
\usepackage{geometry}
\usepackage{titling}
\usepackage{pgfplots}
\setlength{\topskip}{0mm}
\setlength{\droptitle}{-8em} 
\title{{\large \textbf{CONCORDIA UNIVERSITY \\ DEPARTMENT OF COMPUTER SCIENCE AND SOFTWARE ENGINEERING \\ SOEN 6011: SOFTWARE ENGINEERING PROCESS: SUMMER 2019 \\ DELIVERABLE 1: OPEN PROBLEMS}}}
\author{\normalsize \textbf {STUDENT NAME: MANUSHREE MALLARAJU } \\ \normalsize \textbf{STUDENT IDENTIFICATION NUMBER: 40082236 }}
\date{}
\begin{document}
\maketitle
\section*{{\normalsize PROBLEM T1-OP5. Source(s): “Give a brief description, not exceeding one page, of your function, including the domain and co-domain of function, and the characteristics that make it unique. To ensure that you have attained sufficient background, Test 1 will have a problem related to your function.” }}

\section*{\normalsize \textbf{INTRODUCTION}}
The exponential function \( ab^x \)is one of the power rules of math. As the name of an exponential function is described, it involves an exponent. This exponent is represented with a variable rather than a constant, and its base is represented with constant value rather than a variable. Let (\ f(x) = ab^x \) be an exponential function where “b” is its change factor (or a constant), the exponent “x” is the independent variable (or input of the function), the coefficient “a” is called the initial value of the function (or the y-intercept), and “f(x)” represent the dependent variable (or output of the function)

\section*{\normalsize \textbf{DOMAIN:}}
The domain is a set of all real numbers,R. where :\( b > 0 \) ,\( x > 0 \)

\section*{\normalsize \textbf{CO DOMAIN}}
The co-domain is also a set of all real numbers, R.

\section*{\normalsize \textbf{CHARACTERISTICS}}
1.Exponential functions defined by an equation of the form $ab^x$ are called exponential decay functions if the change factor “b” (fixed base value) is \( 0<b<1 \), or it is also called exponential growth functions if the change factor is \( b > 0 \) \newline
2. the y-intercept is (0,a) and it is located at the initial value “a”. There is no x-intercept. The domain for an exponential decay function of this form is all real numbers and the range is \( y > 0 \)
\section*{\normalsize \textbf{GRAPH}}

\begin{center}
\graphicspath{ {./Function6/}}
\section*{\normalsize \textbf{Representation of $ab^x$}}
 
\includegraphics[width=80 mm]{IMG_1640.jpg}
\includegraphics[width=80 mm]{IMG_1641.jpg}
\end{center}
\end{document}